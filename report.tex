\documentclass[a4paper]{report}
\usepackage{unifrrr}
\usepackage{graphicx}
\usepackage[latin1]{inputenc}
\usepackage{float}
\usepackage[colorlinks=false, pdfborder={0 0 0}]{hyperref}
\usepackage{verbatim}
\usepackage{hyperref}
\usepackage{cite}
\usepackage{bibgerm}
\usepackage[english]{babel}
\usepackage{listings}
\usepackage{color}
\usepackage{appendix}
\usepackage{minitoc}
\definecolor{javared}{rgb}{0.6,0,0} % for strings
\definecolor{javagreen}{rgb}{0.25,0.5,0.35} % comments
\definecolor{javapurple}{rgb}{0.5,0,0.35} % keywords
\definecolor{javadocblue}{rgb}{0.25,0.35,0.75} % javadoc


\lstset{language=Java,
basicstyle=\ttfamily\footnotesize,
keywordstyle=\color{javapurple}\bfseries,
stringstyle=\color{javared},
commentstyle=\color{javagreen},
morecomment=[s][\color{javadocblue}]{/**}{*/},
numbers=left,
numberstyle=\tiny\color{black},
stepnumber=2,
numbersep=10pt,
tabsize=4,
showspaces=false,
showstringspaces=false,
captionpos=b}

\begingroup
    \catcode `\@ = 11
    \catcode `\~ = 13
    \catcode `\% = 12
    \protected\long\gdef\cmt@remove#1%~{\endgroup}
    \ifdefined~
        \global\let\cmt@old~
    \else
        \global\let\cmt@old\relax
    \fi
    \protected\gdef~{\begingroup\catcode`%=12
        \futurelet\next\cmt@}
    \protected\gdef\cmt@
      {\ifx%\next
           \expandafter\cmt@remove
       \else
           \endgroup\expandafter\cmt@old
       \fi}
\endgroup


\setcounter{secnumdepth}{4}
\setcounter{tocdepth}{4}

%--------------------------------------------------------------------


%The body of the LaTeX file
\begin{document}  



%Including of the title page. See titlepage.tex file
%Starting the title page. A \begin command always ends with a \end command
\begin{titlepage} 
	%Center all the following stuff
	\begin{center}
		
		%Include the unifr.jpg file from ./images. \\ is a line break
		\includegraphics[scale=0.35]{images/unifr_new.jpg}\\
		
		%Do a vertical space of 0.5 cm
		\vspace{0.5cm}
		
		
	
		\vspace{2cm}
		
		\begin{Large}
		Master Thesis\\
		\end{Large}
		
		\vspace{2cm}
		
		%Start a huge font
		\begin{huge}
			%Sans serif
			{\sf \bf Crowdsourced Product Descriptions and Price Estimations}
		\end{huge}
		
				
		\vspace{2cm}
		
		 Steve Aschwanden\\
		 Dammstrasse 4\\
		 CH-2540 Grenchen\\
		 steve.aschwanden@students.unibe.ch\\
		 05-480-686\\
		
		\vspace{1.5cm}
		
		{\bf Supervisor}\\
		Dr. Gianluca Demartini\\
		C302, Bd de P�rolles 90\\
		CH-1700 Fribourg\\
		demartini@exascale.info\\
		\vspace{2.5cm}
		
		
		Grenchen, \today\\
		
				
	\end{center}
\end{titlepage}


\pagenumbering{arabic}
\pagestyle{plain}
\newpage

\chapter*{Declaration}
\thispagestyle{empty}

\vspace{0.2in}
I declare that this written submission represents my ideas in my own words and where others' ideas or words have been included, I have adequately cited and referenced the original sources. I also declare that I have adhered to all the principles of academic honesty and integrity and have not misrepresented or fabricated or falsified any \mbox{idea/data/fact/source} in my submission. I understand that any violation of the above will be cause for disciplinary action by the Institute and can also evoke penal action from the sources which have thus not been properly cited or from whom proper permission has not been taken when needed.
\vspace{1in}
%\begin{table*}[hb]
\begin{center}
\begin{tabular*}{\textwidth}{@{\extracolsep{\fill}} lr }
\vspace{0.5in}
& Steve Aschwanden, 05-480-686\\
Grenchen; \today: & \hrulefill \\
& (Signature)\\
\end{tabular*}
\end{center}
\chapter*{Acknowledgements}
\thispagestyle{empty}
First of all I thank Dr. Gianluca Demartini for the possibility to write my thesis based on a self-defined topic and, for the support and the ideas he gave me during this time.\\
Furthermore, I express gratitude to the eXascale Infolab\footnote{http://www.exascale.info} for giving me the opportunity to execute my experiments and for the helpful remarks after the mid-term presentation.\\
I thank my fellow student Marcel for the cooperation over the three years of study. Also for the tips and inputs during the thesis.\\
I thank my family which always believe in me. A full-time study wasn't possible without you. 
\clearpage

\chapter*{Abstract}
\thispagestyle{empty}
The creation of auctions for the online marketplace eBay is time consuming and repetitive. The first step for selling an item is taking pictures of it. To complete the auction, the user has to provide a title, description, category and other default parameters. One of the most important step is the definition of a starting price.\\\\
Crowdsourcing is used to generate the required information for a complete auction based on several images. The complex task is split into multiple subtasks. The thesis presents a pure and a hybrid crowdsourcing approach. Different experiments were made to investigate the behaviour of the crowd.\\\\
A promised commission for successful auctions has the biggest influence to the quality of the workers. A majority favours the results of this experiment over the descriptions of the corresponding real online auction. The workers did the most accurate price predictions if the actual market price of the items has been provided.\\\\  
The results of the executed experiments show the potential of the crowd. If all the strength of the single variations will be combined and the task design improved slightly then the generated contents can be used to create real auctions on eBay in the future.
\clearpage


\dominitoc
\tableofcontents
\newpage

\listoffigures

\listoftables

\lstlistoflistings

\chapter{Introduction}
\minitoc

\section{Introduction}

The number of new tweets per second on the social media platform Twitter\footnote{https://www.twitter.com} is huge (over hundred thousand per second). To handle such a big amount of data, a scalable, fault-tolerant real-time framework has to be used. Storm was benchmarked at processing one million 100 byte messages per second per node. It allows to classify every new tweet, for example with a fuzzy logic approach. The so called ''Spout'' is responsible for fetching the tweets and the ''Bolts'' can do some transformation on the received data or persist them in some sort of storage. Cassandra\footnote{http://cassandra.apache.org} is an ideal solution for the given scenario, because it is a distributed, elastically scalable, highly available, fault-tolerant and column-oriented database.

\subsection{Problem Statement}

\subsubsection{Research Questions}\label{research_questions}
\vspace{0.5cm}
\begin{itemize}
\item How can fuzzy classifications be applied to Twitter feeds using Storm?\cite{platemate}
\item How can the results of such classifications be stored in a Cassandra column store?
\end{itemize}

\clearpage

\selectlanguage{english}
\bibliographystyle{bib/custom}
\bibliography{bib/references}
\newpage
\begin{appendices}
\chapter{Some Appendix}


\section{README}
\lstset{caption={}}
\begin{lstlisting}
Fuzzily classify twitter messages using storm and store to cassandra
===


Setup Cassandra (on ubuntu):
---
1. Make sure oracle JDK is installed (1.6+): https://help.ubuntu.com/community/Java#Oracle_Java_7
2. Add the DataStax repository key to your aptitude trusted keys.
> $ curl -L http://debian.datastax.com/debian/repo_key | sudo apt-key add -
3. Install Cassandra:
> sudo apt-get update && sudo apt-get install cassandra
4. Create keyspace and tables:
> cqlsh
> run commands from src/main/resources/createDatabase.txt

Build Runnable jar
---
1. Open a terminal window, navigate to pom.xml directory (project root)
2. Execute the following command:
> mvn clean compile assembly:single
3. In target/, a runnable jar tsfc.jar is created

Run Program
---
> java -jar tsfc.jar <<comma separated list of topics to watch (without whitespace)>>
\end{lstlisting}

\end{appendices}

\end{document}
