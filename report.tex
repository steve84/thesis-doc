\documentclass[a4paper]{report}
\usepackage{unifrrr}
\usepackage{graphicx}
\usepackage[latin1]{inputenc}
\usepackage{float}
\usepackage[colorlinks=false, pdfborder={0 0 0}]{hyperref}
\usepackage{verbatim}
\usepackage{hyperref}
\usepackage{cite}
\usepackage{bibgerm}
\usepackage[english]{babel}
\usepackage{listings}
\usepackage{color}
\usepackage{appendix}
\usepackage{minitoc}
\definecolor{javared}{rgb}{0.6,0,0} % for strings
\definecolor{javagreen}{rgb}{0.25,0.5,0.35} % comments
\definecolor{javapurple}{rgb}{0.5,0,0.35} % keywords
\definecolor{javadocblue}{rgb}{0.25,0.35,0.75} % javadoc


\lstset{language=Java,
basicstyle=\ttfamily\footnotesize,
keywordstyle=\color{javapurple}\bfseries,
stringstyle=\color{javared},
commentstyle=\color{javagreen},
morecomment=[s][\color{javadocblue}]{/**}{*/},
numbers=left,
numberstyle=\tiny\color{black},
stepnumber=2,
numbersep=10pt,
tabsize=4,
showspaces=false,
showstringspaces=false,
captionpos=b}

\begingroup
    \catcode `\@ = 11
    \catcode `\~ = 13
    \catcode `\% = 12
    \protected\long\gdef\cmt@remove#1%~{\endgroup}
    \ifdefined~
        \global\let\cmt@old~
    \else
        \global\let\cmt@old\relax
    \fi
    \protected\gdef~{\begingroup\catcode`%=12
        \futurelet\next\cmt@}
    \protected\gdef\cmt@
      {\ifx%\next
           \expandafter\cmt@remove
       \else
           \endgroup\expandafter\cmt@old
       \fi}
\endgroup


\setcounter{secnumdepth}{4}
\setcounter{tocdepth}{4}

%--------------------------------------------------------------------


%The body of the LaTeX file
\begin{document}  



%Including of the title page. See titlepage.tex file
%Starting the title page. A \begin command always ends with a \end command
\begin{titlepage} 
	%Center all the following stuff
	\begin{center}
		
		%Include the unifr.jpg file from ./images. \\ is a line break
		\includegraphics[scale=0.35]{images/unifr_new.jpg}\\
		
		%Do a vertical space of 0.5 cm
		\vspace{0.5cm}
		
		
	
		\vspace{2cm}
		
		\begin{Large}
		Master Thesis\\
		\end{Large}
		
		\vspace{2cm}
		
		%Start a huge font
		\begin{huge}
			%Sans serif
			{\sf \bf Crowdsourced Product Descriptions and Price Estimations}
		\end{huge}
		
				
		\vspace{2cm}
		
		 Steve Aschwanden\\
		 Dammstrasse 4\\
		 CH-2540 Grenchen\\
		 steve.aschwanden@students.unibe.ch\\
		 05-480-686\\
		
		\vspace{1.5cm}
		
		{\bf Supervisor}\\
		Dr. Gianluca Demartini\\
		C302, Bd de P�rolles 90\\
		CH-1700 Fribourg\\
		demartini@exascale.info\\
		\vspace{2.5cm}
		
		
		Grenchen, \today\\
		
				
	\end{center}
\end{titlepage}


\pagenumbering{arabic}
\pagestyle{plain}
\newpage

\chapter*{Declaration}
\thispagestyle{empty}

\vspace{0.2in}
I declare that this written submission represents my ideas in my own words and where others' ideas or words have been included, I have adequately cited and referenced the original sources. I also declare that I have adhered to all the principles of academic honesty and integrity and have not misrepresented or fabricated or falsified any \mbox{idea/data/fact/source} in my submission. I understand that any violation of the above will be cause for disciplinary action by the Institute and can also evoke penal action from the sources which have thus not been properly cited or from whom proper permission has not been taken when needed.
\vspace{1in}
%\begin{table*}[hb]
\begin{center}
\begin{tabular*}{\textwidth}{@{\extracolsep{\fill}} lr }
\vspace{0.5in}
& Steve Aschwanden, 05-480-686\\
Grenchen; \today: & \hrulefill \\
& (Signature)\\
\end{tabular*}
\end{center}
\chapter*{Acknowledgements}
\thispagestyle{empty}
First of all I thank Dr. Gianluca Demartini for the possibility to write my thesis based on a self-defined topic and, for the support and the ideas he gave me during this time.\\
Furthermore, I express gratitude to the eXascale Infolab\footnote{http://www.exascale.info} for giving me the opportunity to execute my experiments and for the helpful remarks after the mid-term presentation.\\
I thank my fellow student Marcel for the cooperation over the three years of study. Also for the tips and inputs during the thesis.\\
I thank my family which always believe in me. A full-time study wasn't possible without you. 
\clearpage

\chapter*{Abstract}
\thispagestyle{empty}
The creation of auctions for the online marketplace eBay is time consuming and repetitive. The first step for selling an item is taking pictures of it. To complete the auction, the user has to provide a title, description, category and other default parameters. One of the most important step is the definition of a starting price.\\\\
Crowdsourcing is used to generate the required information for a complete auction based on several images. The complex task is split into multiple subtasks. The thesis presents a pure and a hybrid crowdsourcing approach. Different experiments were made to investigate the behaviour of the crowd.\\\\
A promised commission for successful auctions has the biggest influence to the quality of the workers. A majority favours the results of this experiment over the descriptions of the corresponding real online auction. The workers did the most accurate price predictions if the actual market price of the items has been provided.\\\\  
The results of the executed experiments show the potential of the crowd. If all the strength of the single variations will be combined and the task design improved slightly then the generated contents can be used to create real auctions on eBay in the future.
\clearpage


\dominitoc
\tableofcontents
\newpage

\listoffigures

\listoftables

\lstlistoflistings

\chapter{Introduction}
\minitoc

\section{Statement of the problem}
The first step of creating an online auction is mostly to take pictures of the corresponding item. This help the buyers to get information about the state and quality of the article. After that the item needs a short and clear description, some properties (category, state) and a starting offer. If the seller wants to create a lot of different auctions, the whole procedure is time consuming and boring. A price estimation of an article can be difficult, because the background knowledge is missing and other auctions to compare aren't available at any time. Machines aren't able to solve all these steps by themself, because the spectrum of the articles is huge and image processing methodes aren't capable to classify them all correctly. To get all the needed parts of an online auction, a human powered approach is necessary. Crowdsourcing platforms provide the possibility to solve tasks, which are difficult to handle for a computer. 

\section{Existing research}

\section{Goals and objectives}
The thesis will have the following goals and their corresponding objectives:
\begin{itemize}
	\item \textbf{Collect auction item properties by the crowd}
	\begin{itemize}
		\item Analyze the composition of an auction item and select the parts which can be crowdsourced
		\item Form a ground truth including different auctions created by real online auction platform users
		\item Design and publish tasks to gather data from the crowd
		\item Evaluate the quality of the generated content
	\end{itemize}
	\item \textbf{Try to improve the initial solution by implementing a hyprid approach}
	\begin{itemize}
		\item Search for image processing methodes which can simplify and/or support a human intelligence task
		\item Implement the methodes and adapt the task design
		\item Publish the new tasks on the same crowdsourcing platform
		\item Evalutate the results and compare them to the first solution 
	\end{itemize}
\end{itemize}

If the main goals of the thesis are fulfilled, some optional goals can be covered by the thesis:
\begin{itemize}
	\item \textbf{Implement a web application which combines the created subtasks to a complete workflow}
	\begin{itemize}
		\item Find a web application framework and a crowdsourcing platform which provide APIs in the same programming language
		\item Create a workflow which put all the subtasks together to an overall solution
		\item The user can manage the items (upload pictures to create new items, edit and remove items) and directly create an online auction
	\end{itemize}
\end{itemize}

\section{Evaluation}

\clearpage

\selectlanguage{english}
\bibliographystyle{bib/custom}
\bibliography{bib/references}
\newpage
\begin{appendices}
\chapter{Some Appendix}


\section{README}
\lstset{caption={}}
\begin{lstlisting}
Fuzzily classify twitter messages using storm and store to cassandra
===


Setup Cassandra (on ubuntu):
---
1. Make sure oracle JDK is installed (1.6+): https://help.ubuntu.com/community/Java#Oracle_Java_7
2. Add the DataStax repository key to your aptitude trusted keys.
> $ curl -L http://debian.datastax.com/debian/repo_key | sudo apt-key add -
3. Install Cassandra:
> sudo apt-get update && sudo apt-get install cassandra
4. Create keyspace and tables:
> cqlsh
> run commands from src/main/resources/createDatabase.txt

Build Runnable jar
---
1. Open a terminal window, navigate to pom.xml directory (project root)
2. Execute the following command:
> mvn clean compile assembly:single
3. In target/, a runnable jar tsfc.jar is created

Run Program
---
> java -jar tsfc.jar <<comma separated list of topics to watch (without whitespace)>>
\end{lstlisting}

\end{appendices}

\end{document}
