The hope of an additional reward in form of a commission leads to the best overall result. A majority of the crowd favours the created contents over the ground truth. This setting guides to longer item descriptions and a more precise end price prediction.

A supplementary set of images has no influence to the quality of the auction descriptions. This scenario directs only to longer titles and descriptions. The workers spent more time to answer the questions on average but this additional effort isn't enough to affect the quality of the results.

The price estimations are more accurate if the workers have access to the market price of the items. This experiment has the lowest root mean squared error. A combination of a promised bonus and an available sales price could drive to the best result in the future.

A non-branded item results in a shorter title and a higher price prediction error. But the workers used twice as much time as for branded items.

The crowd predicts the end price of an Applie iPhone more precisely than the implemented machine learning approaches. They can find current auctions on eBay to compare with and can detect actual trends. The machine learning algorithms have to be retrained from time to time because eBay is a real marketplace and the prices will develop over time. Another drawback of the hybrid idea is the specialisation in specific item categories, a certain smartphone model for example. The price advantage of the implementation is negligible in comparison with the accuracy of the crowd.

The human-based approach finds the most suitable eBay category for the items, but the workers weren't able to transfer the correct name to MTurk (Spelling errors). The hybrid solution can be done for free but the resulting category isn't as precise as the human-powered implementation.
