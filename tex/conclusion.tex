\section{Future work}
\subsection{Google Reverse Image Search}
Some tests during the first phase of the thesis has shown that Google's reverse image search doesn't return reliable results for all types of items. If the search algorithm findd websites which contain information about the item, then the application should extract the relevant facts in a certain way. Some experiments should be done and the possibilities of the API should be investigated in the future. 
\subsection{Main Image Selection}
The workers have to decide which of the available images give the best r\'{e}sum\'{e} of the product. If the application should publish the auction on eBay in the future, then a representing image of the item has to be determined. 
\subsection{Fully Automated Application}
The creation, administration and evaluation of every subtask is done manually at the moment. The final product can be a mobile application and/or a web service which manages the whole process. The user will take some pictures of the item and upload the data to the server. Then, he has to provide some auction specific information (duration, shipping details) and the software will create the auction after all missing inputs are generated by the crowd. Different pricing strategies can be selected. The estimated price can be used as the starting price or the price can be reduced by a predefined percentage rate. 
\subsection{Price Estimation Game}
Another idea to estimate the starting price is inspired by a German TV game show. The candidate has to predict the cost of an article. After the first guess, the game master answers with `higher' or `lower' until the right guess occurs or the time is running out. If the player finds the correct price, then she/he will win the object.  
The idea of the show is modified to implement a game with a purpose similar to the ESP game project \cite{esp}. The general procedure of the game is the following: 
\begin{enumerate}
	\item The system waits until two independent players are connected and ready to play. 
	\item A few pictures, title and description of the article are displayed and the players had to read them first. 
	\item Then, the game starts and a first guess of the price will be shown by the system. 
	\item Both users have to decide if the real price is higher or lower than the displayed one. 
	\item Dependent on the previous response, the system will present a higher or lower price until the countdown is expired or there are no guesses left. 
	\item The players will receive a score dependent on the difference of the price estimation. A smaller difference leads to a higher score, a higher one to a lower score. 
\end{enumerate}
The first guess of the system will be the mean value \( \mu \) of a large number of sold items on eBay. The value can be determined by the eBay API. The guessing structure will be implemented as a directed binary tree. The root node represents the mean value and every following child node will have a lower (left child) \( v_l \) or higher (right child) value  \( v_r \) determined by the value of the parent node  \( v_p \) and the depth  \( d \) of the tree. The following formula calculates the values  of the nodes: 
\begin{equation}
v_l(v_p,d) = v_p - \frac{\mu}{2^d}
\end{equation}
\begin{equation}
v_r(v_p,d) = v_p + \frac{\mu}{2^d}
\end{equation}
The leafs are integer values which can't be divided by two and represents the final guess of a player. If the time is up and the guesser doesn't reach a leaf node, the value of the actual node is taken. 
The score of the price prediction is determined by a scoring function \( s \) where \( x_1 \) and \( x_2 \) are the price estimations of player 1 and 2.
\begin{equation}
s(x_1,x_2) = 1 - |\varphi(x_1) - \varphi(x_2)|
\end{equation}
The function \( \varphi \) is responsible to normalise the estimations (interval from 0 to 1).
\begin{equation}
\varphi(x) = \frac{x}{2\mu}
\end{equation}
The function is also used to weight the different estimations for the same product. If \( n \) rounds were played for a given object, the final price \( t \) will be calculated:
\begin{equation}
t = \frac{1}{\sum_{k=1}^{n} s(x_{k1},x_{k2})}\left(\sum_{i=1}^{n} s(x_{i1},x_{i2})\frac{x_{i1}+x_{i2}}{2}\right)
\end{equation}
The reliability \( r \) of the price estimation is the mean score of all played games for the same object:
\begin{equation}
r = \frac{1}{n}\left(\sum_{i=1}^{n} s(x_{i1},x_{i2})\right)
\end{equation}
The formulas of the presented idea are the results of a brainstorming and have to be proven first.
\section{Pros and Cons}
This section states the assets and drawbacks of the thesis idea from the viewpoint of the author.
\subsection{Pros}
Workers are impartial and enumerate the facts of the item based on the pictures. They don't try to write a sales text. The provided photos aren't ideal to describe the peculiarities of the item. The worker has no chance to mention the storage size of a smartphone if no screenshot of the system properties is given. If the photographer follows some guidelines to catch the properties of the item, then the workers can deliver good work. The participants of the price task have fun to guess the most accurate price. This fact has a positive influence on the prediction results. 
\subsection{Cons}
Workers are lazy and minimalist. They copy item descriptions from the websites of the producers and take the first found price of sale for the estimation. The prior goal is to maximise the hourly wage and not to commit high quality content. Only the owner of the items knows about small defects and the peculiarities of the item. This important information is missing in the final description and doesn't allow to create an accurate public sale. The spelling of the written descriptions is improvable.
